\section{Gewöhnliche Differentialgleichungen}

\subsection{Dynamische Systeme}

\begin{Definition}[Vektorfeld]
Ein Vektorfeld ist eine Abbildung $$v : \Omega \subset \mathbb{R}^n \to \mathbb{R}^n \; ,$$ die jedem Punkt $x  \in \Omega$ einen Vektor $v(x) \in \mathbb{R}^n$ zuordnet.
\end{Definition}

\begin{Definition}[Integralkurve eines Vertorfeldes]
Eine Weg $\varphi : I \subset \mathbb{R}^n$ heißt Integralkurve in dem Vektorfeld $v : \Omega \subset \mathbb{R}^n \to \mathbb{R}^n$, falls 
$$\varphi' (t) = v(\varphi(t))$$ gilt für alle $t \in  I$.
\end{Definition}

\begin{Definition}[Dynamisches System]
Ein dynamisches System ist eine  Abbildung $F : U \subset \mathbb{R} \times \mathbb{R}^n \to \mathbb{R}^n$, die jedem Punkt $(t,x)  \in U$ einen Vektor $F(t,x) \in \mathbb{R}^n$ zuordnet. Eine Integralkurve oder Lösung für $F$ ist eine Weg $\varphi : I \to \mathbb{R}^n$ mit 
$$\varphi'(t) = F(t, \varphi(t)) $$
für alles $t \in I$.
\end{Definition}


\begin{Satz}[Gewöhnliche Differentialgleichung und Afangswertproblem]
\end{Satz}

\subsubsection*{Normen}
Wir betrachten in diesem Abschnitt immer die Norm $|| x || := \max_i | x_i |$. 

\begin{Definition}[Lipschitz-Stetig]
Eine  Abbildung $F : U \subset \mathbb{R} \times \mathbb{R}^n \to \mathbb{R}^n$ heißt Lipschitz-Stetig,
falls es eine Konstante $L \geq 0$ gibt  mit
$$ || F(t,x) - F(t,x') ||  \leq L || x -x' ||  $$
für alle $(t,x)$ und $(t,x')$ in $U$.
\end{Definition}

\begin{Definition}[Integral vektorwertiger Funktionen]
Für eine  vektorwertige Funktion  $f : I   \to \mathbb{R}^n; f(t) : = \begin{pmatrix} f_1(t)  \\ \vdots \\ f_n(t) \end{pmatrix}$ definieren wir das Integral komponentenweise durch
$$\int_{a}^{b}  f(t) dt := \begin{pmatrix} \int_{a}^{b}  f_1(t) dt  \\ \vdots \\ \int_{a}^{b}  f_n(t) dt \end{pmatrix} \; .$$
\end{Definition}


\begin{Satz}[Integralversion einer Differentialgleichung]

\end{Satz}



\begin{Satz}[Banachscher Fixpunktsatz]
\end{Satz}


\begin{Satz}[Lokaler Existenzsatz (Picard-Lindelöf)]
Das dynamisches System  $$F : U \subset \mathbb{R} \times \mathbb{R}^n \to \mathbb{R}^n$$ sei Lipschitz-Stetig. 
Dann gibt es zu jedem Punkt $(t_0, x_0) \in U$ ein Intervall $I_\delta (t_0) := (t_0 - \delta, t_0 + \delta) \subset \mathbb{R}$ auf dem das AWP 
$$ x' = F(t,x), \; \; x(t_0) = x_0$$
eine Lösung besitzt.
\end{Satz}
\begin{proof}
Es Sei $Q := \bar{I}_a(t_0) \times \bar{K}_b(x_0)$ ein kompakter Quader in $U$ mit Kantenlängen $a$ bzw. $b$.
\end{proof}


\begin{Lemma}[Gronwall]
\end{Lemma}


\begin{Satz}[Eindeutigkeitssatz]
\end{Satz}


\subsection{Lineare Differentialgleichung}

\begin{Definition}
Eine Differentialgleichung der Form
$$ x' (t): = A(t) x(t) + b(t)$$
mit $A: I \subset \mathbb{R} \to \mathbb{R}^{n \times n}$ und $b: I \subset \mathbb{R} \to \mathbb{R}^{n}$ heißt lineare (gewöhnliche) Differentialgleichung.
\end{Definition}

\begin{Satz}[Existenz und Eindeutigkeit]
Ist $ x' (t): = A(t) x(t) + b(t)$ eine lineare Differentialgleichung und $A$ und $b$ stetig, so besitzt das AWP 
$$ x' (t): = A(t) x(t) + b(t) ; \; \; x(t_0) = x_0 $$
genau eine auf ganz $I$ definierte Lösung.
\end{Satz}

\begin{proof}
$F(t,x):= A(t) x(t) + b(t)$ ist Lipschitz-Stetig mit Konstanten $L:= \max_{t \in J}|| A(t) ||$ für jedes kompakte Intervall $J \subset I$.
\end{proof}

\begin{Satz}[Lösungsraum der homogenen Gleichung]
\hfill
\begin{itemize}
\item Die Menge $\mathcal{L}$ der auf $I$ definierten Lösungen der homogenen Gleichung $x'(t) = A(t)x(t)$ ist eine $n$-dimensionaler reeller Vektorraum.
\item $n$ Lösungen $\varphi_1, \cdots, \varphi_n : I \to \mathbb{R}^n$ bilden genau dann eine Basis für $\mathcal{L}$, wenn die Vektoren $\varphi_1(t), \cdots, \varphi_n(t)$ für ein $t \in I$ eine Basis des $\mathbb{R}^n$ bilden.
\end{itemize}
\end{Satz}
\begin{proof}
Sind  $\varphi_1, \cdots, \varphi_n$ Lösungen der homogenen Gleichung, so auch $ c_1 \cdot \varphi_1 + \cdots + c_n \cdot \varphi_n$, da die Ableitung linear ist.
$\mathcal{L}$ ist somit ein Vektorraum. Definiere 
\begin{align*}
& \alpha_{t_0} : \mathcal{L} \to \mathbb{R}^n \\
& \alpha_{t_0} (\varphi) := \varphi(t_0) \; .
\end{align*} 
Aufgrund des Existenzsatzes und der linearität ist $ \alpha_{t_0}$ surjektiv und wegen der Eindeutigkeit der Lösung injektiv.
\end{proof}

\begin{Definition}
Eine Basis  $\varphi_1, \cdots, \varphi_n$ des Lösungsraumes $\mathcal{L}$ der homogenen Gleichung $x'(t) = A(t)x(t)$ heißt Fundamentalsystem.
\end{Definition}

\subsubsection{Lineare Differentialgleichung mit konstanter Matrix $A$}

\begin{Definition}
Für eine Matrix $A \in \mathbb{R}^{n \times n}$ definiert man die Exponentialfunktion 
$$  e^{ A}  := \sum_{k= 0}^{\infty} \frac{1}{k!} A^k \; .$$
Es gilt  
$$  (e^{ tA})' = A e^{tA}  \; .$$
\end{Definition}


\begin{Satz}
Für eine Matrix $A$ lautet die Lösung des Anfangswertproblems $x'(t) = Ax(t)$ und $x(0) = x_0$
$$ x(t) = e^{tA} x_0 \;.$$
\end{Satz}