\documentclass[a4paper,13pt]{scrartcl}

\usepackage[utf8]{inputenc}
\usepackage{amsfonts}
\usepackage{amsmath}
\usepackage{amssymb}
\usepackage{amsthm}
\usepackage{color}
\usepackage[ngerman]{babel}
\usepackage[pdftex]{graphicx}
%\usepackage[T1]{fontenc}
\usepackage{graphicx}
\pagestyle{empty}

%\topmargin20mm
\oddsidemargin0mm
\parindent0mm
\parskip2mm
\textheight25.5cm
\textwidth15.8cm
\unitlength1mm



\begin{document}
\section*{\large  Übungsblatt 01 \hfill Angewandte Mathematik }
\hrule
\hrule
\vspace{4mm}
%\includegraphics[width=0.8\textwidth]{sampleplot.pdf}
{\bf Aufgabe 1}

Gegeben ist die Funktion 
\begin{align*}
& f : \mathbb{R}^4 \to \mathbb{R} \\
& f(x,y,z,t) = x^2 + z\cdot t \cdot e^y 
\end{align*}

{\bf a)}  Berechnen Sie den Gradienten im allgemeinen Punkt $(x,y,z,t)^t$.

{\bf b)}  Berechnen Sie den Gradienten im  Punkt $(1,0,1,2)^t$.

{\bf c)}  Berechnen Sie die Hessematrix $f''(x,y,z,t)$ für einen allgemeinen Punkt $(x,y,z,t)^t$.

{\bf d)}  Berechnen Sie die Hessematrix $f''(1,0,1,2)$.


\vspace{8mm}
{\bf Aufgabe 2}

Berechnen Sie für die Funktion 
\begin{align*}
& f : \mathbb{R}^2 \to \mathbb{R} \\
& f(x,y) = x^2  - 2y^2 
\end{align*}
die Richtungsableitung am Punkt $(1,2)$ in Richtung $h:= (\frac{1}{\sqrt{2}}, \frac{1}{\sqrt{2}})$.

\vspace{8mm}
{\bf Aufgabe 3}

Warum ist die Funktion $f(x) = e^{|x|}$ nicht differenzierbar in $0$?


\vspace{8mm}
{\bf Aufgabe 4}

Gegeben ist der Bereich $A:= \{ (x,y) \in \mathbb{R}^2 \; | \;  y \neq 0 \}$ und die Funktion
\begin{align*}
& f : A  \to \mathbb{R} \\
& f(x,y) = \frac{e^x}{y} \;.
\end{align*}
 Berechnen Sie die Taylorreihe zweiter Ordnung für beliebige Punkte $(a_1, a_2) \in A$.

\vspace{8mm}
{\bf Aufgabe 5}

Gegeben ist der Weg 
\begin{align*}
& \gamma : [0, 2 \pi] \to \mathbb{R}^2 \\
& \gamma(t) = (cos(t), sin(t))^t
\end{align*}
und die Funktion 
\begin{align*}
& f : \mathbb{R}^2 \to\mathbb{R}  \\
& f(x,y) = \sqrt{x^2 + y^2} \;.
\end{align*}
Berechnen Sie $\frac{d}{dt} (f \circ \gamma) (t)$ mit und ohne  Baby Kettenregel.

\vspace{8mm}
{\bf Aufgabe 6}

Berechnen Sie für die Funktionen 
\begin{align*}
& f :  \mathbb{R}^3 \to \mathbb{R} \\
& f(x,y,z) = 2x^2 + y^4 + 2z^2 + 4yz
\end{align*}
die kritischen Punkte und untersuchen Sie diese auf lokale Maxima, Minima oder Sattelpunkte.
\vspace{8mm}

{\bf Aufgabe 7}

Gegeben sind  die Funktionen 
\begin{align*}
& f :  \mathbb{R}^2  \to \mathbb{R}^3 \\
& f(u,v) = \begin{pmatrix}  u +v \\ u -v \\ u^2 + v^2 -1  \end{pmatrix}
\end{align*}
und 
\begin{align*}
& g :  \mathbb{R}^3  \to \mathbb{R} \\
& g(x,y,z) = x^2 + y^2 + z^2 \;. 
\end{align*}
Berechnen Sie den Gradienten von $f \circ g$.



\end{document}

