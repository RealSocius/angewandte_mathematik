\documentclass[a4paper,13pt]{scrartcl}

\usepackage[utf8]{inputenc}
\usepackage{amsfonts}
\usepackage{amsmath}
\usepackage{amssymb}
\usepackage{amsthm}
\usepackage{color}
\usepackage[ngerman]{babel}
\usepackage[pdftex]{graphicx}
%\usepackage[T1]{fontenc}
\usepackage{graphicx}
\pagestyle{empty}

%\topmargin20mm
\oddsidemargin0mm
\parindent0mm
\parskip2mm
\textheight25.5cm
\textwidth15.8cm
\unitlength1mm



\begin{document}
\section*{\large  Übungsblatt 02 \hfill Angewandte Mathematik }
\hrule
\hrule
\vspace{4mm}
%\includegraphics[width=0.8\textwidth]{sampleplot.pdf}
{\bf Aufgabe 1}
Berechnen Sie $\int_M f d \mu$ mit 
\begin{itemize}
\item $M := \{ x \in \mathbb{R}^2 \; | \; 0 \leq x_1 \leq x_2 \leq 1 \}$ und $f(x) = \frac{\sin(x_2)}{x_2}$.
\item $M := \{ x \in \mathbb{R}^2 \; | \; 1 \leq x_1  \leq 2; 0 \leq x_2 \leq 2 \}$ und $f(x) =x_1 +  x_2^3$.
\item $M := \{ x \in \mathbb{R}^2 \; | \; 0 \leq x_1  \leq 1; 0 \leq x_2 \leq 3 -2x_1 \}$ und $f(x) =x_1^2 x_2$.
\item $M := \{ x \in \mathbb{R}^2 \; | \;  x_1^2 + x_2^2 \leq 2 x_2\}$ und $f(x) =  x_1^2 + x_2^2$.
\end{itemize}
\vspace{8mm}

{\bf Aufgabe 2}
Integrieren Sie die Funktion $f(x,y) := \sqrt{1 - (x^2 + y^2)}$ über den Einheitskreis indem Sie die kartesischen Koordinaten in Polarkoordinaten transformieren. 


{\bf Aufgabe 3}
Berechnen Sie das Volumen der Menge $M := \{ (x,y,z) \in \mathbb{R}^3 \; | \; x,y,z \geq 0 , x^2 +y^2 \leq 1, x +y+z \leq \sqrt{2}\}$ 


\end{document}

